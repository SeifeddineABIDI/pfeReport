\chapter*{Conclusion générale}
\addcontentsline{toc}{chapter}{Conclusion générale}
\markboth{Conclusion générale}{}

Le présent document synthétise le projet de fin d'études réalisé au sein de l'entreprise Sofrecom Tunisie, intitulé "La refonte d'une chaîne CI/CD pour une application de gestion centralisée de correspondances". Le projet s'est articulé autour de plusieurs phases :\\

La première phase consistait à améliorer la chaîne CI existante en suivant les meilleures pratiques de l'équipe DevOps d'Orange. La deuxième partie impliquait la mise en place d'un mécanisme de rollback vers la version précédente en cas d'échec de déploiement pour le projet "IHMScribeFront". Le troisième chapitre se focalisait sur la séparation des Frontend et Backend du projet "IHMAdminCatalogue" en mettant en place leurs propres chaînes CI/CD.\\

Le chapitre suivant abordait la création d'une nouvelle chaîne CI/CD pour le projet "Provider", tandis que le chapitre subséquent se concentrait sur l'optimisation du temps d'exécution de la chaîne CI/CD pour le projet "Scribe". Enfin, le dernier chapitre avait pour objectif de réorganiser et simplifier le pipeline de PROD/PREPROD pour le projet "Scribe".\\

Cette réalisation a suivi une méthodologie Scrum, et malgré les défis rencontrés en cours de route, nous avons atteint nos objectifs et satisfait aux exigences du projet. Nous avons veillé à ce que notre prototype soit extensible, permettant l'ajout de serveurs de déploiement sans difficultés, en testant le déploiement sur les mêmes serveurs utilisés par l'équipe SCRIBE. De plus, nous avons appliqué avec succès la même automatisation DevOps à un nouveau projet (IHM Scribe) pour démontrer sa généralisabilité.\\

Dans les étapes à venir, nous prévoyons de migrer vers notre solution en déployant notre prototype sur tous les autres groupes de serveurs SCRIBE, à l'exception des groupes "ihm" et "was".\\

Notre stage chez Sofrecom Tunisie a été une expérience exceptionnellement enrichissante. Nous avons eu le privilège de travailler aux côtés d'une équipe collaborative et hautement compétente, ce qui a contribué à notre développement tant sur le plan académique que professionnel.




