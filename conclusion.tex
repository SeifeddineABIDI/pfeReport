\chapter*{Conclusion générale}
\addcontentsline{toc}{chapter}{Conclusion générale}
\markboth{Conclusion générale}{}

Ce projet de fin d’études, réalisé au sein de SIGA, a permis de concevoir et de déployer un « Portail de Gestion des Approbations avec Workflows Dynamiques », répondant à un besoin critique des organisations modernes : simplifier et centraliser la gestion des processus d’approbation. À travers une démarche structurée en sept sprints, nous avons transformé une problématique complexe en une solution intuitive, performante et évolutive, tout en adoptant des technologies et des pratiques alignées sur les standards actuels. \\

Les premiers chapitres ont posé les fondations du projet, en analysant le contexte et les besoins, puis en implémentant les fonctionnalités de base comme l’authentification et la gestion des utilisateurs avec Angular et Spring Boot. Les sprints suivants ont enrichi le portail avec des fonctionnalités avancées, telles que la soumission et le suivi des demandes via des workflows dynamiques orchestrés par Camunda, ainsi que des outils d’analyse pour superviser les activités et améliorer l’expérience utilisateur, notamment grâce à l’intégration d’un chatbot. Le dernier sprint, axé sur les pratiques DevOps et GitOps, a permis d’automatiser le déploiement avec GitHub Actions, ArgoCD, Docker et Kubernetes, garantissant une scalabilité et une fiabilité optimales tout en sécurisant les données sensibles via des secrets Kubernetes. \\

Ce projet a été une opportunité d’apprentissage exceptionnelle, mêlant défis techniques et collaboration étroite avec l’équipe de SIGA. Il a nécessité une maîtrise approfondie des technologies modernes et une adaptation constante aux imprévus, comme la configuration des runners auto-hébergés ou la gestion des synchronisations avec ArgoCD. Ces obstacles, bien que complexes, ont renforcé notre capacité à résoudre des problèmes de manière méthodique et à produire une solution robuste, prête à répondre aux besoins réels des utilisateurs. Au-delà des aspects techniques, cette expérience a également permis de développer des compétences en gestion de projet et en travail d’équipe, essentielles dans un environnement professionnel dynamique. \\

En perspectives, ce portail pourrait évoluer vers une intégration plus poussée avec des outils d’intelligence artificielle pour anticiper les besoins des utilisateurs, ou encore vers une adoption de solutions serverless pour optimiser les coûts et la performance. Une extension à des environnements multi-clusters Kubernetes pourrait également être envisagée pour répondre aux besoins de grandes entreprises.\\

En conclusion, ce projet a atteint ses objectifs initiaux en offrant une solution complète et moderne pour la gestion des approbations, tout en ouvrant la voie à de futures améliorations. Il illustre la puissance des approches DevOps et GitOps dans le développement de systèmes complexes, et témoigne de l’impact transformateur des technologies de l’information lorsqu’elles sont mises au service des besoins organisationnels. Cette expérience au sein de SIGA restera une étape marquante, tant pour les compétences techniques acquises que pour la satisfaction d’avoir contribué à une solution qui simplifie le quotidien des utilisateurs.s