\chapter*{Terminologie}
%\addcontentsline{toc}{chapter}{Liste des abréviations}
\markboth{Liste des abréviations }{}


\textbf{Automatisation} : c’est lorsque un processus est complètement automatisé et ne requiert aucune intervention d’un utilisateur.
\\

\textbf{Environnement de développement}
: n’est pas seulement un IDE (Integrated
Development Environment) comme Eclipse ou Visual Studio mais les librairies,
les fichiers de configuration et les serveurs en font partie aussi.\\

\textbf{Intégration}
: est la combinaison des parties de code source séparées dont le but
est de déterminer comment elles fonctionnent comme un tout.\\

\textbf{Build}
: présente l’ensemble d’activités dont l’objectif est la construction de l’application. Il comprend plusieurs tâches comme :
\begin{itemize}
    \item La mise à jour du code source depuis le gestionnaire de version
    \item La compilation du code source de l’application
    \item La génération des artefacts\\
\end{itemize}

\textbf{Une tâche (Job)}
: est un ensemble d’instructions que le runner doit exécuter. \cite{ref3}
L’ensemble des jobs est défini dans un fichier YAML. Nous pouvons voir en temps réel le résultat d’une tâche afin que les développeurs puissent savoir si le job a été exécuté avec succès ou bien il a échoué.
Un job peut être automatiquement lancé lorsqu’un commit est poussé ou bien manuellement exécuté. En effet, le job manuel peut être utile, par exemple, pour
automatiser un déploiement, mais ne le déployer que lorsque quelqu’un l’approuve manuellement. En plus, il existe des moyens qui limitent le nombre de personnes
pouvant exécuter un job.\\

\textbf{Artefact (Artifact)}
: est généré par un job en cas d’exécution avec succès. Les
utilisateurs peuvent télécharger cet artefact pour le tester ou bien l’utiliser
directement \cite{ref3}.\\

\textbf{Pipeline}
: est un groupe de jobs exécutés par étapes. En effet, tous les jobs d’une même étape sont exécutés en parallèle s’il y a suffisamment de Runners simultanés disponibles \cite{ref4}.
Le pipeline passe à l’étape suivante lorsque les jobs en cours terminent leur exécution sans erreurs. En cas d’échec d’au moins un job, l'exécution de pipeline s'arrête.\\

\textbf{Runner}
 : est une machine virtuelle, un container Docker ou un cluster de containers. Son rôle principal est de personnaliser l’environnement d’exécution du job \cite{ref4}.
En particulier, GitLab Runner est un projet Open Source utilisé dans la fonctionnalité d’intégration continue GitLab. Ce dernier communique avec le Runner
à travers une API pour exécuter les scripts écrits dans la configuration GitLab-CI.
\\

\textbf{GOROCO}
 : permet de récupérer la valeur exacte de la version \cite{ref5}. Il se décompose en:
 \begin{enumerate}
     \item \textbf{G}eneration: la génération du logiciel
     \item \textbf{R}evision: les services pack (le cas d’un nombre impaire implique que la version a encore des Bugs à éliminer pour la révision suivante)
     \item \textbf{C}orrection: les mises à jour
 \end{enumerate}

 
 