\chapter*{Introduction générale}
% to include the introduction to the table of content
\markboth{Introduction générale}{} %To redefine the section page head

À une époque où les technologies de l’information redessinent les contours de notre quotidien, leur influence sur les organisations est devenue incontournable. Depuis janvier 2023, le nombre d’internautes a franchi la barre des 5,181 milliards à l’échelle mondiale, illustrant une adoption massive du numérique. Cette révolution technologique impose aux entreprises de repenser leurs processus pour gagner en agilité et en efficacité. Parmi ces processus, la gestion des approbations – qu’il s’agisse de demandes de congés, d’achats ou de budgets – se distingue par sa complexité. Souvent fragmentée, impliquant plusieurs acteurs et sujette aux erreurs, elle représente un défi que les outils traditionnels peinent à relever.\\\\

C’est dans ce contexte que SIGA, une entreprise spécialisée dans les solutions informatiques innovantes, s’impose comme un acteur clé pour accompagner les organisations vers une transformation numérique réussie. Reconnue pour son expertise, SIGA développe des systèmes qui simplifient les opérations tout en répondant aux exigences d’un monde en constante évolution. Mon projet de fin d’études s’inscrit pleinement dans cette mission : la création d’un « Portail de Gestion des Approbations avec Workflows Dynamiques et Déploiement Automatisé ». Ce portail vise à centraliser et fluidifier la gestion des demandes, en permettant aux utilisateurs de les soumettre via une interface intuitive, de suivre leur progression en temps réel, et de bénéficier d’une validation automatisée grâce à des workflows dynamiques.\\\\

Pour concrétiser cette vision, j’ai mobilisé un éventail de technologies modernes : Angular pour une interface utilisateur conviviale, Spring Boot pour un backend robuste, et Camunda pour orchestrer des processus flexibles et évolutifs. En adoptant une approche DevOps, j’ai intégré un pipeline CI/CD avec Jenkins pour automatiser les tests et le déploiement, tout en utilisant Docker pour la conteneurisation et Kubernetes pour garantir la scalabilité. Ces choix techniques reflètent un double objectif : offrir une expérience utilisateur optimale tout en assurant une solution fiable et maintenable pour les équipes techniques.\\\\

Ce rapport retrace les étapes de cette réalisation au sein de SIGA, structuré en sept chapitres qui suivent la progression du projet, depuis son contexte jusqu’à son déploiement final.\\
\newpage

Le premier chapitre explore le contexte général du projet. Nous présenterons SIGA, ses ambitions, et les spécificités du portail, avant d’analyser la problématique des processus d’approbation et de poser les bases du projet.\\\\

Le deuxième chapitre, correspondant au Sprint 0, se concentre sur l’analyse et la spécification des besoins. Nous définirons les exigences fonctionnelles et non fonctionnelles, ainsi que les bases de la modélisation des workflows.\\\\

Le troisième chapitre, Sprint 1, aborde l’accès et l’administration de base du portail. Nous détaillerons la mise en place des fonctionnalités initiales, telles que l’authentification et la gestion des utilisateurs, en utilisant Angular et Spring Boot.\\\\

Le quatrième chapitre, Sprint 2, est dédié à la gestion des demandes. Nous présenterons la soumission, le suivi et la validation des demandes, en intégrant les workflows dynamiques avec Camunda.\\\\

Le cinquième chapitre, Sprint 3, se focalise sur le suivi et la supervision. Nous explorerons les fonctionnalités permettant aux utilisateurs de suivre les processus, consulter les demandes d’approbation et gérer les crédits de congés.\\\\

Le sixième chapitre, Sprint 4, met l’accent sur l’analyse et les améliorations. Nous décrirons les outils d’analyse avancés, comme la consultation du calendrier d’équipe, des tâches accomplies et des rapports, ainsi que l’intégration d’un chatbot pour assister les administrateurs.\\\\

Le septième chapitre conclut le projet en abordant la partie DevOps et le déploiement. Nous détaillerons l’utilisation de 'ArgoCD' pour l’intégration continue, Docker pour la conteneurisation, et Kubernetes pour l’orchestration, garantissant une scalabilité et une fiabilité optimales, accompagnées de la documentation associée.\\\\

En conclusion de cette introduction, ces sept chapitres forment un récit cohérent qui reflète les différentes étapes de notre travail. Ensemble, ils illustrent comment ce portail, né d’une problématique concrète, a évolué pour devenir une solution complète, alignée sur les besoins des utilisateurs et les standards technologiques actuels. Ce projet au sein de SIGA a été une opportunité unique de conjuguer créativité, rigueur technique et collaboration, dans un effort pour rendre les processus d’approbation plus simples et plus intelligents.